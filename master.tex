\documentclass{llncs}

%  The subcaption package allows for subfloat figure within a single float.
%  This package substitutes the depregated subfigure and subfig packages 
%  allowing to have subfigures within figures, or subtables within table 
%  floats. Subfloats have their own caption, and an optional global 
%  caption. 
%  >> WARNING: some journal templates from Springer and IEETrans might not
%              be compatible with this package forcing to use the 
%              deprecated packages instead.
\usepackage{subcaption}
\usepackage{subfig}

%  The following command loads a graphics package to include images 
%  in the document. It may be necessary to specify a DVI driver option,
%  e.g., [dvips], but that may be inappropriate for some LaTeX 
%  installations. 
\usepackage[]{graphicx}

% In order to include files without having a clear page using \include*, 
% the newclude package is required
\usepackage{newclude}

\usepackage{array}

% Required for acronyms
% use \acresetall to reset the acroyms counter
\usepackage{acro}

% Managing TODOES and unfinished figures
\usepackage{todonotes}

% Package for nice table
\usepackage{booktabs}

% Mathematics extra symols and commands
\usepackage{amssymb, amsmath}
\usepackage{pifont,amsfonts} % import fonts for tick and x-mark
  % define the extra symbols
  \newcommand{\cmarkgLarge}{\text{\large \color{green!60!black!80}\ding{51}}}
  \newcommand{\cmarkrLarge}{\text{\large \color{red!60!black!80}\ding{51}}}
  \newcommand{\xmarkLarge}{\text{\large \color{red!60!black!80}\ding{55}}}
  \newcommand{\cmark}{\text{\color{green!60!black!80}\ding{51}}}
  \newcommand{\xmark}{\text{\color{red!60!black!80}\ding{55}}}

%% In order to draw some graphs
\usepackage{tikz,xifthen}
\usepackage{tikz-qtree}
\usetikzlibrary{decorations.pathmorphing} % noisy shapes
\usetikzlibrary{fit}                                            % fitting shapes to coordinates
\usetikzlibrary{backgrounds}                                    % drawing the background after the foreground
\usetikzlibrary{shapes,arrows,shadows}
\usetikzlibrary{calc,decorations.pathreplacing,decorations.markings,positioning}
\usetikzlibrary{snakes,decorations.text,shapes,patterns}
\usetikzlibrary{snakes}
\usetikzlibrary{decorations}
\usetikzlibrary{decorations.text}
\usetikzlibrary{decorations.markings}
\usetikzlibrary{shapes}
\usetikzlibrary{patterns}
\usepackage{pgfplots}

%%----- To generate stand onle tikz legends
% argument #1: any options
\newenvironment{customlegend}[1][]{%
    \begingroup
    % inits/clears the lists (which might be populated from previous
    % axes):
    \csname pgfplots@init@cleared@structures\endcsname
    \pgfplotsset{#1}%
}{%
    % draws the legend:
    \csname pgfplots@createlegend\endcsname
    \endgroup
}%
% makes \addlegendimage available (typically only available within an
% axis environment):
\def\addlegendimage{\csname pgfplots@addlegendimage\endcsname}
\pgfkeys{/pgfplots/number in legend/.style={%
        /pgfplots/legend image code/.code={%
            \node at (0.295,-0.0225){#1};
        },%
    },
}
%%---- end tikz legends


% Clever cross referencing. Using cleverref, instead of writting 
% figure~\ref{...} or equation~\ref{...}, only \cref{...} is required.
% The package interprates the references and introduces the figure, fig.,
% equation, eq., etc keywords. \Cref forces first letter capital. 
% >> WARNING: This package needs to be loaded after hyperref, math packages,
%             etc. if used.
%             Cleveref is recomended to load late
%\usepackage{hyperref}
\usepackage{cleveref}

% SI units
\usepackage{siunitx}
% Define the money way to write
\sisetup{
  group-four-digits = true,
  group-separator = {,}
}

% Nice package with citeauthor
%\usepackage{natbib}        % contains the latex packages
\title{Classification of \acs*{sdoct} volumes with \acs*{lbp}: application to \acs*{dme} detection}
\titlerunning{Optimization approach to BUS lesion segmentation}  % abbreviated title (for running head)
%                                     also used for the TOC unless
%                                     \toctitle is used
%
\author{
  Guillaume~Lema\^itre\thanks{Corresponding author: \email{guillaume.lemaitre@udg.edu}}\inst{1,2} 
  \and Mojdeh~Rastgoo\inst{1,2}
  \and Joan~Massich\inst{2}
  \and Shrinivasan~Sankar\inst{2}
  \and D\'esir\'e~Sidib\'e\inst{2}
  \and Fabrice~M\'eriaudeau\inst{2}
}
%
\authorrunning{Lema\^itre et al.} % abbreviated author list (for running head)
%
%
\institute{ViCOROB, Universitat de Girona, Campus Montilivi, Edifici P4, \\17071 Girona, Spain,
\and
LE2I UMR6306, CNRS, Arts et M\'etiers, Univ. Bourgogne Franche-Comt\'e, \\12 rue de la Fonderie, 71200 Le Creusot, France
}
%\addtocmark{Hamiltonian Mechanics} % additional mark in the TOC
             % contains the Title and Autor info
%%%%%%%%%%%%%%%%%%%%%%%%%%%%%%%%%%%%%%%%%%%%%%%%%%%%%%%%%%%%% 
%>>>> uncomment following for page numbers
% \pagestyle{plain}    
%>>>> uncomment following to start page numbering at 301 
%\setcounter{page}{301} 
      % contains package and variables init.
%% Acronym definition example using glossaries package
%% \usepackage{acro} is required
%% 
%% For a powerful usage of the acro package look at http://tex.stackexchange.com/questions/135975/how-to-define-an-acronym-by-using-other-acronym-and-print-the-abbreviations-toge

\DeclareAcronym{us}{
  short = US,
  long  = Ultra-Sound
}

\DeclareAcronym{cad}{
  short = CAD,
  long  = Computer Aided Diagnosis
}

\DeclareAcronym{dm}{
  short = DM,
  long  = Digital Mammography
}

\DeclareAcronym{gt}{
  short = GT,
  long  = Ground Truth
}

\DeclareAcronym{bus}{
%  short = B\acs*{us},
%  long  = Breast \acifused{us}{\acs*{us}}{\acl*{us}}
short = BUS,
long= Breast Ultra-Sound
}

\DeclareAcronym{ml}{
  short = ML,
  long  = Machine Learning
}

\DeclareAcronym{acm}{
  short = ACM,
  long  = Active Contour Model
}

\DeclareAcronym{crf}{
  short = CRFs,
  long  = Conditional Random Fields
}

\DeclareAcronym{mrf}{
  short = MRFs,
  long  = Markov Random Fields
}

\DeclareAcronym{cv}{
  short = CV,
  long  = Computer Vision
}
\DeclareAcronym{icm}{
  short = ICM,
  long  = Iterated Conditional Modes
}
\DeclareAcronym{sa}{
  short = SA,
  long  = Simulate Anealing
}
\DeclareAcronym{gc}{
  short = GC,
  long  = Graph-Cuts
}

\DeclareAcronym{aov}{
  short = AOV,
  long  = Area Overlap
}

\DeclareAcronym{birads}{
  short = BI-RADS,
  long  = Breast Imaging-Reporting and Data System
}

\DeclareAcronym{mad}{
  short = MAD,
  long  = Median Absolute Deviation
}

\DeclareAcronym{qc}{
  short = QC,
  long  = Quadratic-Chi
}

\DeclareAcronym{sift}{
  short = SIFT,
  long  = Self-Invariant Feature Transform
}

\DeclareAcronym{bof}{
  short = BoF,
  long  = Back-of-Features
}

\DeclareAcronym{acr}{
  short = ACR,
  long  = American College of Radiology
}

\DeclareAcronym{fa}{
  short = FA,
  long  = Fibro-Adenoma
}

\DeclareAcronym{dic}{
  short = DIC,
  long  = Ductal Inflating Carcinoma
}

\DeclareAcronym{ilc}{
  short = ILC,
  long  = Inflating Lobular Carcinoma
}

\DeclareAcronym{fpr}{
  short = FPR,
  long  = False Positive Ratio
}

\DeclareAcronym{fnr}{
  short = FNR,
  long  = False Negative Ratio
}

\DeclareAcronym{fp}{
  short = FP,
  long  = False Positive
}

\DeclareAcronym{rbf}{
  short = RBF,
  long  = Radial Basis Function
}

\DeclareAcronym{dr}{
  short = DR,
  long  = Diabetic Retinopathy
}

\DeclareAcronym{dme}{
  short = DME,
  long  = Diabetic Macula Edema
}

\DeclareAcronym{oct}{
  short = OCT,
  long  = Optical Coherence Tomography
}

\DeclareAcronym{sdoct}{
  short = SD-OCT,
  long  = Spectral Domain OCT
}

\DeclareAcronym{amd}{
  short = AMD,
  long = Age-related Macular Degeneration
}

\DeclareAcronym{hog}{
  short = HOG,
  long = Histogram of Oriented Gradients
}

\DeclareAcronym{svm}{
  short = SVM,
  long = Support Vector Machines
}

\DeclareAcronym{bow}{
  short = BoW,
  long = Bag-of-Words
}

\DeclareAcronym{rf}{
  short = RF,
  long = Random Forest
}

\DeclareAcronym{roc}{
  short = ROC,
  long = Receiver Operating Characteristic
}

\DeclareAcronym{auc}{
  short = AUC,
  long = Area Under the Curve
}

\DeclareAcronym{lbp}{
  short = LBP,
  long = Local Binary Patterns
}

\DeclareAcronym{pca}{
  short = PCA,
  long = Principal Component Analysis
}

\DeclareAcronym{nlm}{
  short = NL-mean,
  long = Non-Local Mean
}

\DeclareAcronym{lopo}{
  short = LOPO,
  long =  Leave-One-Patient Out
}

\DeclareAcronym{lbptop}{
  short = LBP-TOP,
  long =  Local Binary Pattern histogram from Three Orthogonal Planes
}

\DeclareAcronym{se}{
  short = SE,
  long =  Sensitivity
Planes
}

\DeclareAcronym{sp}{
  short = SP,
  long =  Specificity
Planes
}

\DeclareAcronym{sw}{
  short = SW,
  long =  sliding window 
Planes
}      % contains the acronims 

%% Select inputing only one part of the document
%\includeonly{content/intro/intro}   % the file wihtout .tex
%\includeonly{content/other/other_content}



%
\begin{document}

\maketitle              % typeset the title of the contribution

\begin{abstract}
The abstract should summarize the contents of the paper
using at least 70 and at most 150 words. It will be set in 9-point
font size and be inset 1.0 cm from the right and left margins.
There will be two blank lines before and after the Abstract. \dots
\keywords{computational geometry, graph theory, Hamilton cycles}
\end{abstract}

%
\subsection{Autonomous Systems}
%
In this section, we will consider the case when the Hamiltonian $H(x)$
is autonomous. For the sake of simplicity, we shall also assume that it
is $C^{1}$.

We shall first consider the question of nontriviality, within the
general framework of
$\left(A_{\infty},B_{\infty}\right)$-subquadratic Hamiltonians. In
the second subsection, we shall look into the special case when $H$ is
$\left(0,b_{\infty}\right)$-subquadratic,
and we shall try to derive additional information.
%
\subsubsection{The General Case: Nontriviality.}
%
We assume that $H$ is
$\left(A_{\infty},B_{\infty}\right)$-sub\-qua\-dra\-tic at infinity,
for some constant symmetric matrices $A_{\infty}$ and $B_{\infty}$,
with $B_{\infty}-A_{\infty}$ positive definite. Set:
\begin{eqnarray}
\gamma :&=&{\rm smallest\ eigenvalue\ of}\ \ B_{\infty} - A_{\infty} \\
  \lambda : &=& {\rm largest\ negative\ eigenvalue\ of}\ \
  J \frac{d}{dt} +A_{\infty}\ .
\end{eqnarray}

Theorem~\ref{ghou:pre} tells us that if $\lambda +\gamma < 0$, the
boundary-value problem:
\begin{equation}
\begin{array}{rcl}
  \dot{x}&=&JH' (x)\\
  x(0)&=&x (T)
\end{array}
\end{equation}
has at least one solution
$\overline{x}$, which is found by minimizing the dual
action functional:
\begin{equation}
  \psi (u) = \int_{o}^{T} \left[\frac{1}{2}
  \left(\Lambda_{o}^{-1} u,u\right) + N^{\ast} (-u)\right] dt
\end{equation}
on the range of $\Lambda$, which is a subspace $R (\Lambda)_{L}^{2}$
with finite codimension. Here
\begin{equation}
  N(x) := H(x) - \frac{1}{2} \left(A_{\infty} x,x\right)
\end{equation}
is a convex function, and
\begin{equation}
  N(x) \le \frac{1}{2}
  \left(\left(B_{\infty} - A_{\infty}\right) x,x\right)
  + c\ \ \ \forall x\ .
\end{equation}

%
\begin{proposition}
Assume $H'(0)=0$ and $ H(0)=0$. Set:
\begin{equation}
  \delta := \liminf_{x\to 0} 2 N (x) \left\|x\right\|^{-2}\ .
  \label{eq:one}
\end{equation}

If $\gamma < - \lambda < \delta$,
the solution $\overline{u}$ is non-zero:
\begin{equation}
  \overline{x} (t) \ne 0\ \ \ \forall t\ .
\end{equation}
\end{proposition}
%
\begin{proof}
Condition (\ref{eq:one}) means that, for every
$\delta ' > \delta$, there is some $\varepsilon > 0$ such that
\begin{equation}
  \left\|x\right\| \le \varepsilon \Rightarrow N (x) \le
  \frac{\delta '}{2} \left\|x\right\|^{2}\ .
\end{equation}

It is an exercise in convex analysis, into which we shall not go, to
show that this implies that there is an $\eta > 0$ such that
\begin{equation}
  f\left\|x\right\| \le \eta
  \Rightarrow N^{\ast} (y) \le \frac{1}{2\delta '}
  \left\|y\right\|^{2}\ .
  \label{eq:two}
\end{equation}

\begin{figure}
\vspace{2.5cm}
\caption{This is the caption of the figure displaying a white eagle and
a white horse on a snow field}
\end{figure}

Since $u_{1}$ is a smooth function, we will have
$\left\|hu_{1}\right\|_\infty \le \eta$
for $h$ small enough, and inequality (\ref{eq:two}) will hold,
yielding thereby:
\begin{equation}
  \psi (hu_{1}) \le \frac{h^{2}}{2}
  \frac{1}{\lambda} \left\|u_{1} \right\|_{2}^{2} + \frac{h^{2}}{2}
  \frac{1}{\delta '} \left\|u_{1}\right\|^{2}\ .
\end{equation}

If we choose $\delta '$ close enough to $\delta$, the quantity
$\left(\frac{1}{\lambda} + \frac{1}{\delta '}\right)$
will be negative, and we end up with
\begin{equation}
  \psi (hu_{1}) < 0\ \ \ \ \ {\rm for}\ \ h\ne 0\ \ {\rm small}\ .
\end{equation}

On the other hand, we check directly that $\psi (0) = 0$. This shows
that 0 cannot be a minimizer of $\psi$, not even a local one.
So $\overline{u} \ne 0$ and
$\overline{u} \ne \Lambda_{o}^{-1} (0) = 0$. \qed
\end{proof}
%
\begin{corollary}
Assume $H$ is $C^{2}$ and
$\left(a_{\infty},b_{\infty}\right)$-subquadratic at infinity. Let
$\xi_{1},\allowbreak\dots,\allowbreak\xi_{N}$  be the
equilibria, that is, the solutions of $H' (\xi ) = 0$.
Denote by $\omega_{k}$
the smallest eigenvalue of $H'' \left(\xi_{k}\right)$, and set:
\begin{equation}
  \omega : = {\rm Min\,} \left\{\omega_{1},\dots,\omega_{k}\right\}\ .
\end{equation}
If:
\begin{equation}
  \frac{T}{2\pi} b_{\infty} <
  - E \left[- \frac{T}{2\pi}a_{\infty}\right] <
  \frac{T}{2\pi}\omega
  \label{eq:three}
\end{equation}
then minimization of $\psi$ yields a non-constant $T$-periodic solution
$\overline{x}$.
\end{corollary}
%

We recall once more that by the integer part $E [\alpha ]$ of
$\alpha \in \bbbr$, we mean the $a\in \bbbz$
such that $a< \alpha \le a+1$. For instance,
if we take $a_{\infty} = 0$, Corollary 2 tells
us that $\overline{x}$ exists and is
non-constant provided that:

\begin{equation}
  \frac{T}{2\pi} b_{\infty} < 1 < \frac{T}{2\pi}
\end{equation}
or
\begin{equation}
  T\in \left(\frac{2\pi}{\omega},\frac{2\pi}{b_{\infty}}\right)\ .
  \label{eq:four}
\end{equation}

%
\begin{proof}
The spectrum of $\Lambda$ is $\frac{2\pi}{T} \bbbz +a_{\infty}$. The
largest negative eigenvalue $\lambda$ is given by
$\frac{2\pi}{T}k_{o} +a_{\infty}$,
where
\begin{equation}
  \frac{2\pi}{T}k_{o} + a_{\infty} < 0
  \le \frac{2\pi}{T} (k_{o} +1) + a_{\infty}\ .
\end{equation}
Hence:
\begin{equation}
  k_{o} = E \left[- \frac{T}{2\pi} a_{\infty}\right] \ .
\end{equation}

The condition $\gamma < -\lambda < \delta$ now becomes:
\begin{equation}
  b_{\infty} - a_{\infty} <
  - \frac{2\pi}{T} k_{o} -a_{\infty} < \omega -a_{\infty}
\end{equation}
which is precisely condition (\ref{eq:three}).\qed
\end{proof}
%

\begin{lemma}
Assume that $H$ is $C^{2}$ on $\bbbr^{2n} \setminus \{ 0\}$ and
that $H'' (x)$ is non-de\-gen\-er\-ate for any $x\ne 0$. Then any local
minimizer $\widetilde{x}$ of $\psi$ has minimal period $T$.
\end{lemma}
%
\begin{proof}
We know that $\widetilde{x}$, or
$\widetilde{x} + \xi$ for some constant $\xi
\in \bbbr^{2n}$, is a $T$-periodic solution of the Hamiltonian system:
\begin{equation}
  \dot{x} = JH' (x)\ .
\end{equation}

There is no loss of generality in taking $\xi = 0$. So
$\psi (x) \ge \psi (\widetilde{x} )$
for all $\widetilde{x}$ in some neighbourhood of $x$ in
$W^{1,2} \left(\bbbr / T\bbbz ; \bbbr^{2n}\right)$.

But this index is precisely the index
$i_{T} (\widetilde{x} )$ of the $T$-periodic
solution $\widetilde{x}$ over the interval
$(0,T)$, as defined in Sect.~2.6. So
\begin{equation}
  i_{T} (\widetilde{x} ) = 0\ .
  \label{eq:five}
\end{equation}

Now if $\widetilde{x}$ has a lower period, $T/k$ say,
we would have, by Corollary 31:
\begin{equation}
  i_{T} (\widetilde{x} ) =
  i_{kT/k}(\widetilde{x} ) \ge
  ki_{T/k} (\widetilde{x} ) + k-1 \ge k-1 \ge 1\ .
\end{equation}

This would contradict (\ref{eq:five}), and thus cannot happen.\qed
\end{proof}
%
\paragraph{Notes and Comments.}
The results in this section are a
refined version of \cite{clar:eke};
the minimality result of Proposition
14 was the first of its kind.

To understand the nontriviality conditions, such as the one in formula
(\ref{eq:four}), one may think of a one-parameter family
$x_{T}$, $T\in \left(2\pi\omega^{-1}, 2\pi b_{\infty}^{-1}\right)$
of periodic solutions, $x_{T} (0) = x_{T} (T)$,
with $x_{T}$ going away to infinity when $T\to 2\pi \omega^{-1}$,
which is the period of the linearized system at 0.

\begin{table}
\caption{This is the example table taken out of {\it The
\TeX{}book,} p.\,246}
\begin{center}
\begin{tabular}{r@{\quad}rl}
\hline
\multicolumn{1}{l}{\rule{0pt}{12pt}
                   Year}&\multicolumn{2}{l}{World population}\\[2pt]
\hline\rule{0pt}{12pt}
8000 B.C.  &     5,000,000& \\
  50 A.D.  &   200,000,000& \\
1650 A.D.  &   500,000,000& \\
1945 A.D.  & 2,300,000,000& \\
1980 A.D.  & 4,400,000,000& \\[2pt]
\hline
\end{tabular}
\end{center}
\end{table}
%
\begin{theorem} [Ghoussoub-Preiss]\label{ghou:pre}
Assume $H(t,x)$ is
$(0,\varepsilon )$-subquadratic at
infinity for all $\varepsilon > 0$, and $T$-periodic in $t$
\begin{equation}
  H (t,\cdot )\ \ \ \ \ {\rm is\ convex}\ \ \forall t
\end{equation}
\begin{equation}
  H (\cdot ,x)\ \ \ \ \ {\rm is}\ \ T{\rm -periodic}\ \ \forall x
\end{equation}
\begin{equation}
  H (t,x)\ge n\left(\left\|x\right\|\right)\ \ \ \ \
  {\rm with}\ \ n (s)s^{-1}\to \infty\ \ {\rm as}\ \ s\to \infty
\end{equation}
\begin{equation}
  \forall \varepsilon > 0\ ,\ \ \ \exists c\ :\
  H(t,x) \le \frac{\varepsilon}{2}\left\|x\right\|^{2} + c\ .
\end{equation}

Assume also that $H$ is $C^{2}$, and $H'' (t,x)$ is positive definite
everywhere. Then there is a sequence $x_{k}$, $k\in \bbbn$, of
$kT$-periodic solutions of the system
\begin{equation}
  \dot{x} = JH' (t,x)
\end{equation}
such that, for every $k\in \bbbn$, there is some $p_{o}\in\bbbn$ with:
\begin{equation}
  p\ge p_{o}\Rightarrow x_{pk} \ne x_{k}\ .
\end{equation}
\qed
\end{theorem}
%
\begin{example} [{{\rm External forcing}}]
Consider the system:
\begin{equation}
  \dot{x} = JH' (x) + f(t)
\end{equation}
where the Hamiltonian $H$ is
$\left(0,b_{\infty}\right)$-subquadratic, and the
forcing term is a distribution on the circle:
\begin{equation}
  f = \frac{d}{dt} F + f_{o}\ \ \ \ \
  {\rm with}\ \ F\in L^{2} \left(\bbbr / T\bbbz; \bbbr^{2n}\right)\ ,
\end{equation}
where $f_{o} : = T^{-1}\int_{o}^{T} f (t) dt$. For instance,
\begin{equation}
  f (t) = \sum_{k\in \bbbn} \delta_{k} \xi\ ,
\end{equation}
where $\delta_{k}$ is the Dirac mass at $t= k$ and
$\xi \in \bbbr^{2n}$ is a
constant, fits the prescription. This means that the system
$\dot{x} = JH' (x)$ is being excited by a
series of identical shocks at interval $T$.
\end{example}
%
\begin{definition}
Let $A_{\infty} (t)$ and $B_{\infty} (t)$ be symmetric
operators in $\bbbr^{2n}$, depending continuously on
$t\in [0,T]$, such that
$A_{\infty} (t) \le B_{\infty} (t)$ for all $t$.

A Borelian function
$H: [0,T]\times \bbbr^{2n} \to \bbbr$
is called
$\left(A_{\infty} ,B_{\infty}\right)$-{\it subquadratic at infinity}
if there exists a function $N(t,x)$ such that:
\begin{equation}
  H (t,x) = \frac{1}{2} \left(A_{\infty} (t) x,x\right) + N(t,x)
\end{equation}
\begin{equation}
  \forall t\ ,\ \ \ N(t,x)\ \ \ \ \
  {\rm is\ convex\ with\  respect\  to}\ \ x
\end{equation}
\begin{equation}
  N(t,x) \ge n\left(\left\|x\right\|\right)\ \ \ \ \
  {\rm with}\ \ n(s)s^{-1}\to +\infty\ \ {\rm as}\ \ s\to +\infty
\end{equation}
\begin{equation}
  \exists c\in \bbbr\ :\ \ \ H (t,x) \le
  \frac{1}{2} \left(B_{\infty} (t) x,x\right) + c\ \ \ \forall x\ .
\end{equation}

If $A_{\infty} (t) = a_{\infty} I$ and
$B_{\infty} (t) = b_{\infty} I$, with
$a_{\infty} \le b_{\infty} \in \bbbr$,
we shall say that $H$ is
$\left(a_{\infty},b_{\infty}\right)$-subquadratic
at infinity. As an example, the function
$\left\|x\right\|^{\alpha}$, with
$1\le \alpha < 2$, is $(0,\varepsilon )$-subquadratic at infinity
for every $\varepsilon > 0$. Similarly, the Hamiltonian
\begin{equation}
H (t,x) = \frac{1}{2} k \left\|k\right\|^{2} +\left\|x\right\|^{\alpha}
\end{equation}
is $(k,k+\varepsilon )$-subquadratic for every $\varepsilon > 0$.
Note that, if $k<0$, it is not convex.
\end{definition}
%

\paragraph{Notes and Comments.}
The first results on subharmonics were
obtained by Rabinowitz in \cite{rab}, who showed the existence of
infinitely many subharmonics both in the subquadratic and superquadratic
case, with suitable growth conditions on $H'$. Again the duality
approach enabled Clarke and Ekeland in \cite{clar:eke:2} to treat the
same problem in the convex-subquadratic case, with growth conditions on
$H$ only.

Recently, Michalek and Tarantello (see \cite{mich:tar} and \cite{tar})
have obtained lower bound on the number of subharmonics of period $kT$,
based on symmetry considerations and on pinching estimates, as in
Sect.~5.2 of this article.

%
% ---- Bibliography ----
%
\begin{thebibliography}{5}
%
\bibitem {clar:eke}
Clarke, F., Ekeland, I.:
Nonlinear oscillations and
boundary-value problems for Hamiltonian systems.
Arch. Rat. Mech. Anal. 78, 315--333 (1982)

\bibitem {clar:eke:2}
Clarke, F., Ekeland, I.:
Solutions p\'{e}riodiques, du
p\'{e}riode donn\'{e}e, des \'{e}quations hamiltoniennes.
Note CRAS Paris 287, 1013--1015 (1978)

\bibitem {mich:tar}
Michalek, R., Tarantello, G.:
Subharmonic solutions with prescribed minimal
period for nonautonomous Hamiltonian systems.
J. Diff. Eq. 72, 28--55 (1988)

\bibitem {tar}
Tarantello, G.:
Subharmonic solutions for Hamiltonian
systems via a $\bbbz_{p}$ pseudoindex theory.
Annali di Matematica Pura (to appear)

\bibitem {rab}
Rabinowitz, P.:
On subharmonic solutions of a Hamiltonian system.
Comm. Pure Appl. Math. 33, 609--633 (1980)

\end{thebibliography}

\end{document} 
