\documentclass{llncs}

%% Latex documents that need direct input
%  The subcaption package allows for subfloat figure within a single float.
%  This package substitutes the depregated subfigure and subfig packages 
%  allowing to have subfigures within figures, or subtables within table 
%  floats. Subfloats have their own caption, and an optional global 
%  caption. 
%  >> WARNING: some journal templates from Springer and IEETrans might not
%              be compatible with this package forcing to use the 
%              deprecated packages instead.
\usepackage{subcaption}
\usepackage{subfig}

%  The following command loads a graphics package to include images 
%  in the document. It may be necessary to specify a DVI driver option,
%  e.g., [dvips], but that may be inappropriate for some LaTeX 
%  installations. 
\usepackage[]{graphicx}

% In order to include files without having a clear page using \include*, 
% the newclude package is required
\usepackage{newclude}

% Required for acronyms
% use \acresetall to reset the acroyms counter
\usepackage{acro}

% Managing TODOES and unfinished figures
\usepackage{todonotes}

% Mathematics extra symols and commands
\usepackage{amssymb, amsmath}
\usepackage{pifont,amsfonts} % import fonts for tick and x-mark
  % define the extra symbols
  \newcommand{\cmarkgLarge}{\text{\large \color{green!60!black!80}\ding{51}}}
  \newcommand{\cmarkrLarge}{\text{\large \color{red!60!black!80}\ding{51}}}
  \newcommand{\xmarkLarge}{\text{\large \color{red!60!black!80}\ding{55}}}
  \newcommand{\cmark}{\text{\color{green!60!black!80}\ding{51}}}
  \newcommand{\xmark}{\text{\color{red!60!black!80}\ding{55}}}

%% In order to draw some graphs
\usepackage{tikz,xifthen}
\usepackage{tikz-qtree}
\usetikzlibrary{decorations.pathmorphing} % noisy shapes
\usetikzlibrary{fit}                                            % fitting shapes to coordinates
\usetikzlibrary{backgrounds}                                    % drawing the background after the foreground
\usetikzlibrary{shapes,arrows,shadows}
\usetikzlibrary{calc,decorations.pathreplacing,decorations.markings,positioning}
\usetikzlibrary{snakes,decorations.text,shapes,patterns}
\usetikzlibrary{snakes}
\usetikzlibrary{decorations}
\usetikzlibrary{decorations.text}
\usetikzlibrary{decorations.markings}
\usetikzlibrary{shapes}
\usetikzlibrary{patterns}
\usepackage{pgfplots}

%%----- To generate stand onle tikz legends
% argument #1: any options
\newenvironment{customlegend}[1][]{%
    \begingroup
    % inits/clears the lists (which might be populated from previous
    % axes):
    \csname pgfplots@init@cleared@structures\endcsname
    \pgfplotsset{#1}%
}{%
    % draws the legend:
    \csname pgfplots@createlegend\endcsname
    \endgroup
}%
% makes \addlegendimage available (typically only available within an
% axis environment):
\def\addlegendimage{\csname pgfplots@addlegendimage\endcsname}
\pgfkeys{/pgfplots/number in legend/.style={%
        /pgfplots/legend image code/.code={%
            \node at (0.295,-0.0225){#1};
        },%
    },
}
%%---- end tikz legends


% Clever cross referencing. Using cleverref, instead of writting 
% figure~\ref{...} or equation~\ref{...}, only \cref{...} is required.
% The package interprates the references and introduces the figure, fig.,
% equation, eq., etc keywords. \Cref forces first letter capital. 
% >> WARNING: This package needs to be loaded after hyperref, math packages,
%             etc. if used.
%             Cleveref is recomended to load late
%\usepackage{hyperref}
\usepackage{cleveref}
        % contains the latex packages
\title{ An optimization approach to segment breast lesions in ultra-sound images using clinically validated visual cues }
\titlerunning{Optimization approach to BUS lesion segmentation}  % abbreviated title (for running head)
%                                     also used for the TOC unless
%                                     \toctitle is used
%
\author{
  Joan Massich\thanks{This work was partially supported by the Regional 
                      Council of Burgundyand by the Spanish Goverment MEC 
                      grant nb.TIN2012-3171-C02-01}\inst{1} 
  \and Guillaume Lema\^{i}tre\inst{1,2}
  \and Joan Mart\'{i}\inst{2}
  \and Fabrice M\'{eriaudeau}\inst{1}
}
%
\authorrunning{Joan Massich et al.} % abbreviated author list (for running head)
%
%
\institute{LE2I-UMR CNRS 6306, Universit\'{e} de Bourgogne, 12 rue de la Fonderie, \\
  71200 Le Creusot, France;\\
\email{joan.massich@u-bourgogne.fr}
\and
ViCOROB, Universitat de Girona, 
Campus Montilivi, Edifici P4, \\17071 Girona, Spain
}
%\addtocmark{Hamiltonian Mechanics} % additional mark in the TOC
             % contains the Title and Autor info
%%%%%%%%%%%%%%%%%%%%%%%%%%%%%%%%%%%%%%%%%%%%%%%%%%%%%%%%%%%%% 
%>>>> uncomment following for page numbers
% \pagestyle{plain}    
%>>>> uncomment following to start page numbering at 301 
%\setcounter{page}{301} 
      % contains package and variables init.
%% Acronym definition example using glossaries package
%% \usepackage{acro} is required
%% 
%% For a powerful usage of the acro package look at http://tex.stackexchange.com/questions/135975/how-to-define-an-acronym-by-using-other-acronym-and-print-the-abbreviations-toge

\DeclareAcronym{us}{
  short = US,
  long  = Ultra-Sound
}

\DeclareAcronym{cad}{
  short = CAD,
  long  = Computer Aided Diagnosis
}

\DeclareAcronym{dm}{
  short = DM,
  long  = Digital Mammography
}

\DeclareAcronym{gt}{
  short = GT,
  long  = Ground Truth
}

\DeclareAcronym{bus}{
%  short = B\acs*{us},
%  long  = Breast \acifused{us}{\acs*{us}}{\acl*{us}}
short = BUS,
long= Breast Ultra-Sound
}

\DeclareAcronym{ml}{
  short = ML,
  long  = Machine Learning
}

\DeclareAcronym{svm}{
  short = SVM,
  long  = Support Vector Machines
}

\DeclareAcronym{acm}{
  short = ACM,
  long  = Active Contour Model
}

\DeclareAcronym{crf}{
  short = CRFs,
  long  = Conditional Random Fields
}

\DeclareAcronym{mrf}{
  short = MRFs,
  long  = Markov Random Fields
}

\DeclareAcronym{cv}{
  short = CV,
  long  = Computer Vision
}
\DeclareAcronym{icm}{
  short = ICM,
  long  = Iterated Conditional Modes
}
\DeclareAcronym{sa}{
  short = SA,
  long  = Simulate Anealing
}
\DeclareAcronym{gc}{
  short = GC,
  long  = Graph-Cuts
}

\DeclareAcronym{aov}{
  short = AOV,
  long  = Area Overlap
}

\DeclareAcronym{birads}{
  short = BI-RADS,
  long  = Breast Imaging-Reporting and Data System
}

\DeclareAcronym{mad}{
  short = MAD,
  long  = Median Absolute Deviation
}

\DeclareAcronym{qc}{
  short = QC,
  long  = Quadratic-Chi
}

\DeclareAcronym{sift}{
  short = SIFT,
  long  = Self-Invariant Feature Transform
}

\DeclareAcronym{bof}{
  short = BoF,
  long  = Back-of-Features
}

\DeclareAcronym{acr}{
  short = ACR,
  long  = American College of Radiology
}

\DeclareAcronym{fa}{
  short = FA,
  long  = Fibro-Adenoma
}

\DeclareAcronym{dic}{
  short = DIC,
  long  = Ductal Inflating Carcinoma
}

\DeclareAcronym{ilc}{
  short = ILC,
  long  = Inflating Lobular Carcinoma
}

\DeclareAcronym{fpr}{
  short = FPR,
  long  = False Positive Ratio
}

\DeclareAcronym{fnr}{
  short = FNR,
  long  = False Negative Ratio
}

\DeclareAcronym{fp}{
  short = FP,
  long  = False Positive
}

\DeclareAcronym{rbf}{
  short = RBF,
  long  = Radial Basis Function
}

      % contains the acronims 

%% Select inputing only one part of the document
%\includeonly{content/intro/intro}   % the file wihtout .tex
%\includeonly{content/other/other_content}
 
\begin{document} 
\maketitle 

\begin{abstract}
This paper addresses the problem of automatic classification of \ac{sdoct} data for automatic identification of patients with \ac{dme} versus normal subjects.
Our method is based on \ac{lbp} features to describe the texture of \ac{oct} images and we compare different \ac{lbp} features extraction approaches to compute a single signature for the whole \ac{oct} volume.
Experimental results with two datasets of respectively 32 and 30 \ac{oct} volumes show that ...(TB BE WRITTEN BASED ON RESULTS)

Moreover, the experiments show that the proposed method achieves better classification performances than other recent published works.

\keywords{\acl{dme}, \acl{oct}, \acs{dme}, \acs{oct}, \ac{lbp}.}
\end{abstract}

%% Incldue the content without .tex extension
\acresetall  % reset the acronyms from the abstract
% include the figures path relative to the master file
\graphicspath{ {./content/intro/figures/} }

\section{Introduction}
\label{sec:intro}  % \label{} allows reference to this section


Breast cancer is the second most common cancer (1.4 million cases per year, 10.9\% of  diagnosed cancers) after lung cancer, followed by colorectal, stomach, prostate and liver cancers. %~\cite{Ferlay2010}.
In terms of mortality, breast cancer is the fifth most common cause of cancer death.
However, it is ranked as the leading cause of cancer death among females in both western countries and economically developing countries~\cite{cancerStatistics2011}.

Medical imaging plays an important role in breast cancer mortality reduction, contributing to its early detection through screening for diagnosis, image-guided biopsy, treatment follow-up and suchlike procedures~\cite{smith2003american}.
Although \ac{dm} remains the reference imaging modality for breast cancer screening, \ac{us} imaging has proven to be a successful adjunct image modality~\cite{smith2003american}.%,berg2004diagnostic}.
The main advantage of \ac{us} imaging, opposed to other image modalities, lies in the discriminative power that \ac{us} offers to visually differentiate benign from malignant solid lesions~\cite{Stavros:1995p12392}.
In this manner, \ac{us} screening contributes to reduce the amount of unnecessary biopsies~\cite{ciatto1994contribution}, which is estimated to be between $65\sim85\%$ of the prescribed biopsies~\cite{yuan2010multimodality}, in favour of a less traumatic short-term screening follow-up using \ac{bus} images. %~\cite{gordon1995malignant}.
As the standard for assessing this \ac{bus} images, the \ac{acr} proposes the \ac{birads} lexicon for \ac{bus} images~\cite{biradsus}.
This \ac{us} \ac{birads} lexicon is a set of standard markers that characterizes the lesions encoding the visual cues found in \ac{bus} images and facilitates their analysis.
Further details regarding the \ac{us} \ac{birads} lexicon descriptors proposed by the \ac{acr}, can be found in this document in \cref{sec:method:dterm:feat}, where visual cues of \ac{bus} images and breast structures are discussed to define feature descriptors.

All these facts show the interest in the medical community for incorporating \ac{us} screening as part of the standard procedure in breast screening programs~\cite{biradsus}, which encourages the development of \ac{cad} systems using \ac{us} to be applied to breast cancer diagnosis.
Building \ac{cad} systems based on the clinical tools already in use (\emph{i.e.} \ac{us} \ac{birads} lexicon) is not straight forward.
Shortcomings like the location and explicit delineation of the lesions need to be addressed, since those tasks are intrinsically carried out by the radiologists during their visual assessment of the images to infer the lexicon representation of the lesions.
Therefore, developing accurate segmentation methodologies for breast lesions and structures are crucial in order to develop \ac{cad} systems that can take advantage of the already validated clinical tools.

%Regardless of the clinical utility of the \ac{us} images, such image modality suffers from different inconveniences due to strong noise natural of \ac{us} imaging  and the presence of strong \ac{us} artifacts, both degrading the overall image quality~\cite{Ensminger:2008p6920} which compromise the performance of the radiologists.
%Radiologists infer health state of the patients based on visual inspection of images which by means of some screening technique (e.g.~\ac{us}) depict physical properties of the screened body.
%The radiologic diagnosis error rates are similar to those found in any other tasks requiring human visual inspection, and such errors, are subject to the quality of the images and the ability of the reader to interpret the physical properties depicted on them\cite{manning2005perception}.
%
%Therefore the interest from the medical imaging community, also for the specific case of breast lesion assessment using \ac{us} data, in developing \ac{cad} systems that provide better instrumentation to improve image interpretation, and consequently achieve better diagnosis.

This article proposes a highly modular and flexible framework for segmenting lesions and tissues present in \ac{bus} images.
The proposal takes advantage of an energy-based strategy to perform segmentations based on discrete optimizations using super-pixels and a set of novel features analogous to the elements encoded by the \ac{us} \ac{birads} lexicon~\cite{biradsus}.

%%% Local Variables: 
%%% mode: latex
%%% TeX-master: "../../master.tex"
%%% End: \section{introduction}
          % the file wihtout .tex
% include the figures path relative to the master file
% \graphicspath{ {./content/method/figures/visual_cues/}{./content/method/figures/}}
% \graphicspath{ {./content/method/figures/}}

\section{Materials and Methods}

  This section offers a general description of the methodology proposed for OCT volume classification, whereas further details of some elements involved in the methodology are found as subsections.
  
  The proposed method, as well as, its experimental set-up are outlined in Fig.\,\ref{fig:ML-scheme}.
  The methodology is formulated as a standard classification procedure.
  The available dataset with its acompaining \textcolor{red}{GT} are divided into training $(S1,l1)$ and testing $(S2,l2)$. 
  The final goal is to represent $S1$ and $S2$ in the feature space $F$ by supplying $(sxF,l1)$ as a training to a classifier, using the trained classifier to estimate $l2$ from $S2xF$ and comparing the estimation with the \textcolor{red}{GT}.
   To do so, the images forming the OCT volumes are preprocessed using non-local means (NL-means) algorithm \cite{buades2005non}. This algorithm preserve important details and textures of the original image, while reducing the noise.
   The mapping stage is used to determine a discrete set of elements (or structures) $Z$  which is used for representing the volume $s  in S$.
   The feature detection stage correspond to measurements done in $G(Z)$ used for representing $s$ in terms of $ZxG$. 
   This mapping and feature detection steps can be found as a single-steps in the literature.
   The feature extraction procedure combines the elements in $Z$ and its measurements $G(Z)$ to create the final feature space $F$ and project $s$ on it.
   
   The design choices are all illustrated in Fig.\,\ref{fig:ML-scheme} and discussed further in this section. The work here presented does not discuss in detail neither the mapping, nor the adopted classifier, further than this lines.
   As a possible mappings, for representing the volumes, 2D image slices of the volume and \color{red}{7x7x7}\color{black} sliding volumes, have been considered. 
   As a classifier, a \color{red}{Random Forest}\color{black} using 100 trees, has been considered.
   
% \begin{figure}[h]
% \centering{
%   \includegraphics[width=1\textwidth]{mm.pdf}}
%   \caption{Machine learning classification basic scheme}
%   \label{fig:ML-scheme}
% \end{figure}

\subsection{Data}
\color{red}{
\begin{itemize}
  \item cross-validation
  \item our dataset
  \item DUC dataset
\end{itemize}}\color{black}

For evaluation purposes, the results have been cross-validated, by splitting the data in training and testing using a \color{red}{loo}\color{black} strategy. In this manner for each round a pair \color{red}{dce,normal} has been selected to be used as the round test set, while the rest of the dataset has been used as a training. \color{red}{Doing the cross validation in this manner, has the limitation that despite the fact that the results are robust due to the cross validation, no results variance can be reported. However, and despite this limitation, LOO has been choose due to the reduced amount of OCT volumes available.}\color{black}

\color{red}{The dataset blablablabal...}\color{black}
\color{red}{The duc dataset blabla bla...}\color{black}

\subsection{Image pre-processing}
The pre-processing stage in the proposed methodology applies an image denoising method to reduce the speckle noise in OCT images. Since image details and texture of the original image are needed by the following stages in the method, non-local means (NL-means) algorithm \cite{buades2005non} is used. NL-means algotithm has the advantage to use all the possible self-predictions that the image can provide \cite{buades2005non} rather than local or frequency filters such as Gaussian, anisotropic or Wiener filters. \color{red}{Figure .. shows an OCT slice before and after denoising}\color{black}

\subsection{Feature detection}
Why texture
\subsubsection{LBP}
\subsubsection{LBPTOP}

\subsection{Feature extraction}
\begin{itemize}
  \item explanation of why feat-extraction
  \item Histograms concatenation
  \item PCA(Histgrams concatenation)
\end{itemize}

\color{blue}
%\subsection{Image denoising}
%In order to preserve the details and the texture of the original image, non-local means (NL-means) algorithm \cite{buades2005non} is used. NL-means algotithm has the advantage to use all the possible self-predictions that the image can provide \cite{buades2005non} rather than local or frequency filtes such as Gaussian, anisotropic or Wiener filters. Given the noisy image $\upsilon = \lbrace \upsilon(i) \vert  i\in I\rbrace$, the estimated value for a pixel $i$, is computed as a weighted average of all the pixels in the image, 
%\[ NL[\upsilon](i) = \sum_{j\in I}  w(i,j) \upsilon(j),\]
%The family of the wights $\lbrace w(i,j)\rbrace j$ depend on the Euclidean distance of the intenisty gray level vectors of the two pixels $i$ and $j$. These wights are defined as,  
%\[ w(i,j) = \frac{1}{Z(i)} e^{-\frac{\Vert u(N_{i} - N_{j}) \Vert^2_{2,a}}{h^{2}}}\]
%Where $a > 0$ is the standard deviation of the Gaussian kernel, $Z(i)$ is the normalizing factor and $h$ acts as a degree of filtering \cite{buades2005non}.
%\[Z(i) = \sum_{j} e^{-\frac{\Vert \upsilon(N_{i}) - \upsilon(N_{i}) \Vert^{2}_{2,a}}{h^{2}}}\]

%Figure {\color{red} reference to fig} \color{black} shows one slide before and after denoising.


\subsection{Features extraction and representation}
We are extracting the 2D and 3D LBP features in low and high level manners. The aforementioned approaches are describe in the following.


\subsubsection{Low-level features} - are extracted considering the whole volume using LBP and LBP-TOP descriptors. LBP is a discriminative rotation invariant feature descriptor proposed by Ojala et al. \cite{ojala2002multiresolution}. In this descriptor a central pixel ($g_c$) in a defined neighborhood by $R$ radius is compared to its neighborhood pixels ($g_{p}$, with distance of $R$ from the central pixel) and their differences are encoded in terms of binary patterns (see Eq. \ref{Eq:LBP}). The binary patterns are calculated for each pixel in the given image and their histogram, defines the final descriptor.

	\begin{equation} \label{Eq:LBP}
 		LBP_{P,R} = \sum_{p=0}^{P-1}s(g_{p} - g_{c})2^{p}
	\end{equation}
In this research we consider rotation invariant features with uniform patterns. The uniform patterns are defined by the unifromity measure ($U$) of 2. Uniformity measure corresponds to the number of spatial transitions in the LBP pattern \cite{ojala2002multiresolution}. For instance patterns $00000000_{2}$ and $11111111_{2}$ have the U value of 0 while patterns like $01111111_{2}$ and $00000011_{2}$ have two transitions of 0/1 in their pattern, therfore they are considered as uniform patterns (see Eq. \ref{Eq:LBPru}). 
	\begin{equation}\label{Eq:LBPru}
		LBP_{P,R}^{riu2} = 
  			\begin{cases}
    				 \sum_{p=0}^{P-1}s(g_{p} - g_{c}) & \text{ if } U(LBP_{P,R}) \le 2\\
    				  P+1 & \text{otherwise,}
  			\end{cases}
    \end{equation}
The LBP features are extracted from each slice of the volume and their histograms are concatenated to build the first low-level descriptor. The second low-level descriptor is defined in a similar manner as the first one. However principal component analysis (PCA) is applied to the concatenated histograms in order to reduce the dimensions. The first and second low-level descriptors are obtained using the 2D LBP descriptor. However the third low-level feature is obtained using 3D-LBP (LBP-TOP). Zhao et al. \cite{zhao2007dynamic} proposed Local Binary Pattern histogram from Three Orthogonal Planes (LBP-TOP) as a dynamic texture descriptor. This descriptor is an extension to normal LBP while it considers texture descriptors along the temporal domain. LBP-TOP considers the LBP pattern in three orthogonal planes (see Fig. \ref{fig:LBPTOP-framework}), XY, XT and YT. The obtained LBP patterns from the three planes are concatenated to form the final descriptor. The three low-level descriptors are calculated with the $P$ number of 8, 16 and 24 for the radius if 1, 2 and 3 respectively

\begin{figure}
\centering{
	\includegraphics[width=0.40\textwidth]{./content/method/figures/LBPTOP_fig.png}}
	\caption{LBP-TOP framework, the image is taken from \cite{JiangEtAl13}}
	\label{fig:LBPTOP-framework}
\end{figure}

 
% In the proposed framework we consider different neighborhoods ($P$) of 8, 16, 32 and 64 pixels with radius of 1, 2, 3 and 4 respectively.

%%% VLBP 
% Zhao et al. \cite{zhao2007dynamic} first proposed to model the dynamic texture with Volume Local Binary Pattern (VLBP) or 3D-LBP. VLBP considers the joint distribution of the gray level pixels in a local neighborhood in a monochrome dynamic texture sequence. Figure \ref{fig:VLBP-framework} illustrates the procedure of VLBP for a texture volume. The final length of VLBP feature vector, as it is shown in Fig.\ref{fig:VLBP-framework}, depends on the number of neighboring points ($P$). As this number increases, the final dimension of the feature vector increase by $2^{3P+2}$. Due to this rapid increase, adapting VLBP for large number of neighboring points will be complex.

% \begin{figure}[h]
% \centering{
% 	\includegraphics[width=1\textwidth]{figures/VLBP_fig_h.png}}
% 	\caption{VLBP framework}
% 	\label{fig:VLBP-framework}
% \end{figure}

\subsubsection{High-level features} - are extracted using bag of features (BoF) approach which is illustrated in Fig. \ref{fig:BoF-framework}. After de-noising the images by non-local-mean approach, a patch detection step is carried out by identifying informative regions of the data. These patches can either be sampled densely or sparsely. The dense sampling extracts more information regarding the object appearance. However, it might retain redundant features. In the contrary sparse sampling is based on detecting salient key points from the most informative regions of the volume. Our proposed BoF approach is based on the first strategy. In this stage 2D-LBP (7$\times$7) and 3D-LBP-TOP patches (7$\times$7$\times$7) are extracted for each patient. Defining $N$ as the number of selected patches for each volume and $d$ as the number of feature dimensions, then each volume is characterized by a $N \times d$ feature matrix (see Fig. \ref{fig:BoF-framework}, ``feature extraction''). The next step in bag of features consists of building the dictionary of ``visual-words''. All the feature matrices of the training set are concatenated together and k-means clustering method (see Fig.\ref{fig:BoF-framework}, ``clustering'') is used to define the ``visual-words''. K-means is an iterative algorithm which finds k centroids by alternating assignment and update steps. The assignment steps is based on $L_{2}$ norm (Euclidean) distance. Different initialization methods can be used in order to assign the initial k clusters \cite {celebi2013comparative}, here the initial k clusters are selected based on greedy k-means++ method \cite{arthur2007k}. \textbf{Depending on the framework and application, different choices of the number of ``visual-words'' (number of $k$ clusters) can be made. In our framework, the number of clusters are varying in the range of [2 4 8 16 32 64 100]. SHOULD CHNAGE} Finally, the probability distribution (e.g., histogram) of the ``visual-words'' of each feature matrix is computed (feature quantization) and is used to feed the classifier to be trained. In the prediction stage, the histogram of the feature matrix corresponding to the new volume is computed using the previously learned dictionary and, finally, it is classified by the previously trained classifier.\\

%\tikzstyle{block} = [rectangle, draw, fill=gray!20, text = black,
    text width=6em, text centered, rounded corners, minimum height=4em , minimum width = 6em]
    % \tikzstyle{line} = [draw, -latex']
  \tikzstyle{myarrow}=[->, thick]
    \tikzstyle{line}=[-, thick]
    \tikzstyle{block2} = [rectangle, draw, fill=white!20,
    text width=6em, text centered, rounded corners, minimum height=4em, minimum width = 6em]
    \tikzstyle{block3} = [rectangle, draw, fill=gray!20, text = black,
    text width=7em, text centered, rounded corners, minimum height=4em , minimum width = 7em]
\def\blockdist{1}
\def\edgedist{1.5}
  %%%% The Framework Sparse Coding 

\begin{figure}
 \begin{center}
   \begin{tikzpicture}[node distance = 1cm,scale=0.6, every node/.style={scale=0.6}]
%(FEx.east|- FEx.south)
    \node [block2] (input) {Training image};
    %\node [block, right of = input, node distance = 2.8cm](Seg){Segmentation}; 
    \node [block, right of=input,node distance = 2.8cm](De){Denoising};
    \node [block, right of=De,node distance = 2.8cm](FEx){Feature extraction};
    \path (FEx.east)+(+0.8,0) node (g) {};
    
    %%% Sparse Coding Block
    \node [block3, right of=g,node distance = 1.7cm](DL){Dictionary learning /k-means};
    \node [block3, below of=DL,node distance = 2.5cm](PR){Projection};
    \begin{pgfonlayer}{background}
      \path (DL.west |- DL.north)+(-0.4,-0.1+\blockdist) node (a) {};
      \path (PR.east |- PR.south)+(+0.4,-0.7) node (b) {};          
      \path[fill=gray!10,rounded corners, draw=gray!20, dashed] (a) rectangle (b);
    \end{pgfonlayer}
\path (DL.west |- DL.north)+(+1.2,-0.5+\blockdist) node (SP) {\textbf{Bag of Features}};
\path (PR.east |- PR.south)+(-1.3,-0.4+\blockdist) node (c){};
\path (PR.east)+(-3.15,0) node (d) {};

%%% Testing 
\node [block, below of=FEx, node distance = 2.5cm](FE2){Feature extraction};
\node [block, below of=De, node distance = 2.5cm](De2){Denoising};
% \node [block, below of=Seg, node distance = 2.5cm](Seg2){Segmentation}; 
\node [block2, below of=input, node distance = 2.5cm](TestImg){Testing image};

%%% 
\node [block, right of=PR, node distance = 3.6cm](Pool){Visual words histogram};
\path (Pool.east) + (0.3,0) node (f){}; 
\path (Pool.east) + (0.2,-0.1) node (f1){}; 

%%% Classification
\node [block, right of = Pool, node distance = 3.5cm] (Pre){Prediction}; 
    \node [block, above of = Pre, node distance = 2.5cm] (Learn){Learning}; 
    \begin{pgfonlayer}{background}
      \path (Learn.west |- Learn.north)+(-0.4,-0.1+\blockdist) node (h) {};
    \path (Pre.east |- Pre.south)+(+0.4,-0.7) node (i) {};          
    \path[fill=gray!10,rounded corners, draw=gray!20, dashed] (h) rectangle (i);
\end{pgfonlayer}
\path (Learn.west |- Learn.north)+(+1.1,-0.5+\blockdist) node (Clas) {\textbf{Classification}};
\path (Pre.east |- Pre.south)+(-1.3,-0.4+\blockdist) node (j){};
\path (f1.north)+(0, 2.5) node (k) {};
\path (Pre.east) + (1.2,0) node (k1) {P(..)}; 

    % Draw edges
    \draw [line] (input) -- (De) -- (FEx); 
    \draw [myarrow] (FEx)-- (DL);
    \draw [myarrow] (DL) -- (PR) ; 
    \draw [line] (TestImg) -- (De2) -- (FE2); 
    \draw [myarrow] (FE2) -- (PR) ;
    \draw [line] (PR) -- (Pool); 
    \draw [myarrow] (Pool) -- (Pre); 
    \draw [line] (f1.north) -- + (0,2.5)(k.south); 
    \draw [myarrow] (k.south)+ (0,0.1)  -- (Learn.west); 
    \draw [myarrow] (Pre) -- (k1);

    \end{tikzpicture}
    \end{center}
    

\caption{Bag of features framework} 
\label{fig:BoF-framework}

\end{figure}

\subsection{Classification}

Random Forest is an ensemble of decision trees and was introduced by \cite{breiman2001random}.
The ensemble uses each tree to predict an output and finalize the ultimate
prediction by aggregating the outputs of all tress. This classifier learns the
data by training multiple decision trees on bootstrap samples of the original
data. Each bootstrap of D dimension is used for training one decision tree
and at each node, the best split among randomly (d << D) selected subset
of descriptors is chosen. Each tree is grown to its maximum length without
any pruning. In the prediction stage a sample is voted by each tree and it is
labeled by considering the majority of the votes.

\color{black}

%%% Local Variables: 
%%% mode: latex
%%% TeX-master: "../../master"
%%% End: 
 
% % include the figures path relative to the master file
% \graphicspath{ {./content/results/figures/} }

\section{Experiments}
\subsection{Datasets}
\subsubsection{SERI} - dataset contains 32 OCT volumes (16 DME and 16 normal). This dataset was acquired in \textbf{Institutional Review Board-approved protocols} using CIRRUS TM (Carl Zeiss Meditec, Inc, Dublin, CA) SD-OCT device. All SD-OCT images are read and assessed by trained graders and identifies as normal or DME cases based on evaluation of retinal thickening, hard exudates, intraretinal cystoid space formation and subretinal fluid. This dataset was acquired by our colleagues from Singapore Eye Research Institute (SERI). 

\subsubsection{Duke} - dataset published by Srinivasan et al. \cite{Srinivasan2014}, consists of 45 OCT volumes (15 AMD, 15 DME and 15 normal). All the SD-OCT volumes were acquired in Institutional Review Board-approved protocols using Spectralis SD-OCT (Heidelberg Engineering Ins., Heidelberg, Germany) imaging at Duke University, Harvard University and Michigan University. In this study we only consider a subset of the original data containing 15 DME and 15 normal OCT volumes.

\subsection{Evaluation}
The SERI dataset is provided in complete SD-OCT volumes by 512$\times$1024$\times$128 dimensions. Using this dataset, first the three low-level features such as \textit{LBP}, \textit{LBP+PCA} and \textit{LBP-TOP} are extracted. These descriptors are calculated with the $P$ number of 8, 16 and 24 for the radius if 1, 2 and 3 respectively. The features are classified using RF with 100 tress. The relative results are shown in Tab. \ref{tab:LbPTopVolumeResult}. 

The second experiment is carried using high-level features and BoW approach.  
%----------

%%% Local Variables: 
%%% mode: latex
%%% TeX-master: "../../master"
%%% End: 

% include the figures path relative to the master file
\graphicspath{ {./content/results/figures/} }
\definecolor{acm}{rgb}{0.8 0.7254901960784313 0.4549019607843137}
\newcommand{\acmColor}{yellow}

\definecolor{ml}{rgb}
{0.3333333333333333  0.6588235294117647 0.40784313725490196}
\newcommand{\mlColor}{green}

\definecolor{other}{rgb}
{0.5058823529411764  0.4470588235294118 0.6980392156862745}
\newcommand{\otherColor}{purple}

\definecolor{db}{rgb}
{0.7686274509803922  0.3058823529411765 0.3215686274509804}
\newcommand{\dbColor}{red}

\definecolor{aov}{rgb}
{0.2980392156862745  0.4470588235294118 0.6901960784313725}
\newcommand{\aovColor}{blue}

\tikzstyle{acmStyle} = [fill=acm]
\tikzstyle{mlStyle} = [fill=ml]
\tikzstyle{otherStyle} = [fill=other]
\tikzstyle{nodeBase} = [ rectangle,rounded corners=2pt]





\section{Method evaluation and comparison} 
A 16 \ac{bus} images dataset with accompanying multi-label \ac{gt} delineating all the structures present in the images has been used to evaluate the proposed methodology for lesion segmentation application.
Every image in the dataset presents a single lesion with variable extension. 
The size of the lesions ranges from under $1/100$ to over $1/5$ of the image.
The dataset composed of cysts, \acp{fa}, \acp{dic} and \acp{ilc}.

\Cref{fig:results} shows qualitative results, whereas the quantitative results from the best configuration are reported as a table in \cref{fig:surveyResults:method}. %~\cite{massich2013phd}.
Notice that for the proposed framework, the performance in terms of \ac{aov} is limited by the capacity of the super-pixels to snap the desired boundary.
\Cref{fig:results} shows how the delineation resulting from a proper labeling of the super-pixels differs from the \ac{gt}.
%Despite the fact that a \ac{fpr} of $0.4$ seems significant, based on our experiments most of the images produce no \ac{fp} lesions.
%However, images with \ac{fp} lesions, are likely to produce more several of them as shown in \cref{fig:results}c-e.
%The amount of \ac{fp} lesions can be trimmed by applying a higher cost in the pairwise therm (compare \cref{fig:resuts:smallPWterm} and \cref{fig:results:bigPWterm}.
%Nevertheless, a severe increasing of the pairwise cost also increases the \ac{fnr} since some lesions are missed due to over-smoothing. %~\cite{massich2013phd}.
%The situation of having a larger \ac{fnr} is less desirable than reducing the \ac{fpr}. 
%The \ac{fnr} reported in \cref{fig:surveyResults:method} is caused by an image within the dataset that its lesion is fully contained in a single super-pixel and still around $20\%$ of this super-pixel's area is healty tissue. 

\begin{figure}[t]
  \centering
  
  \begin{tikzpicture}
  \tikzstyle{noMargin} = [inner sep=0mm, outer sep=0mm]
  \node[noMargin](a){
    \includegraphics[trim= 20 0 30 0, clip, height=3cm]{goodQSorigin}
  };
  \node[noMargin, right= 3pt of a](b){
    \includegraphics[trim= 20 0 30 0, clip, height=3cm]{goodQSseg}
    };
  \node[noMargin, right= 3pt of b](c){
    \includegraphics[trim = 0 90 0 0, clip, height=3cm]{fporigin}
  };
  \node[noMargin, right= 3pt of c]{
    \begin{tikzpicture}
      \node[noMargin](d){
      \includegraphics[trim = 0 90 0 0, clip, height=1.2cm]{fpnohom}
      };
      \node[noMargin, below= 5pt of d]{
      \includegraphics[trim = 0 90 0 0, clip, height=1.2cm]{fpHom}
      };
    \end{tikzpicture}
  };

  \end{tikzpicture}

  \caption{\footnotesize Qualitative results. 
    (a) Example 1: orignal image, super-pixels' delineations and \ac{gt}. 
    (b) Differences between \ac{gt} and the delination resulting from super-pixels' boundary.
    (c) Ex. 2.
    (d) weak $V(\cdot,\cdot)$
    (e) strong $V(\cdot,\cdot)$
    \vspace{-10pt}
    }
  \label{fig:results}
\end{figure}

Due to the lack of publicly available data (and code) there is no manner to perform a methodology comparison further than compiling the results reported in the literature.
\Cref{fig:surveyResults} resumes some of the methodologies found in the literature and traslates the reported results into \ac{aov} for comparison purposes.
%compiles relevant metholodologies relevant methodologies found in the literature 
%The table in \Cref{fig:surveyResults:survey} complies the evaluation reported by the authors of the most relevant methodologies found in the literature. %~\cite{massich2013phd}.
%Details about the methodologies proposed in the literature can also be found in the aforesaid table.
%Specifically it is detailed the category of technique been used for detecting the lesions, segmenting it and post-process the delineations (if any).
%The studied categories are: \ac{ml}, \ac{acm} and others.
The iconography used in \cref{fig:surveyResults:survey} relates methodology stages, technology used and if the stages have been treated separately or as a single stage.
%if the those stages are treated as independent and connected in a daisy-chain fashion, or otherwise the stages are addressed in an atomic manner.

\Cref{fig:surveyResults:comparison} arrenge the information for direct visual comparison.
%\Cref{fig:surveyResults:comparison} renders the information present in \cref{fig:surveyResults:survey} and \cref{fig:surveyResults:method} in a visual manner compare all the results at once.
%The methodologies arranged in a radial fashion and grouped by its most representative technology category. 
%In red it can be found a small, medium and large categorization of the dataset reported for testing.
%The concentric circles represent \ac{aov}. 
%The blue line correspond to the \ac{aov} results reported in the literature, whereas the black line indicates the \ac{aov} our framework scored in our testing.
An extra element is also represented in \cref{fig:surveyResults:comparison} as blue swatch delimited by two blue dashed lines.
The boundaries of this swatch correspond to performance of expert radiologists in terms of \ac{aov} based on an inter- and intra-observer experiment carried out by Pons et al.~\cite{gerard2013}.
%\footnote{The dataset used for testing the framework here proposed corresponds to the subset of images used by Pons et al.~\cite{gerard2013} that have accompanying multi-labelled \ac{gt}.}.

When comparing the results it is clear the inconvenience of unexciting public data, since several of the results outperform the manual delineations studied in~\cite{gerard2013}.
It can also be seen that the category tested in larger datasets is \ac{ml}, whereas \ac{acm} lead to better segmentations since the lesion boundary is easier to model in \ac{acm} compared to \ac{ml} based techniques.

\begin{figure}[t]
  \begin{subfigure}[b]{\textwidth}
    {\tiny 
\newcommand{\myCoord}[1]{
  \tikz[remember picture]{\coordinate[remember picture] (#1) at (0,0);
    %\fill[red] (#1) circle[radius=1pt];
  }
}

\newcommand{\drawBox}[5] {
  % draw a colored bar [0..MaxValue] 
  % parameters:
  %  #1 - bar color
  %  #2 - bar width
  %  #3 - bar height
  %  #4 - MaxValue
  %  #5 - actual Value
  \begin{tikzpicture}[baseline=(value.center)]
    \def\w{#2}                   % width of a box
    \def\h{#3}                   % height of a box
    \pgfmathsetmacro{\v}{#5/#4}  % value to display in ratio [0-1]  
    \pgfmathsetmacro{\V}{100*\v} % value to display in percentage   
    \def\x{\v*\w}                % position of the desired value

    % draw rectangle (value dims between color and: white-body, gray-border)
    \draw[draw=#1!\V!gray!] (0,0) rectangle (\w,\h);
    \filldraw[fill=#1!\V!white!, draw=#1!\V!gray!] (0,0) rectangle (\x,\h);
    % display the value
    \path (0,0) -- (0,\h) node[midway,anchor=west] (value) {#5};

%    \draw [blue] (current bounding box.south west) rectangle (current bounding box.north east);
  \end{tikzpicture}
}

\centering
\begin{tabular}{lcccccccccccccccc}
  Method Id\textsuperscript{[ref]}:
              &a\cite{Liu:2010p14328}
              &b\cite{Gao:2012p14336}
              &c\cite{AlemanFlores:2007p14310}
              &d\cite{Huang:2012p14313}
              &e\cite{Madabhushi:2003p6036}
              &f\cite{hao2012combining}
              &g\cite{Zhang:2010p14317}
              &h\cite{Xiao:2002p5639}
              &i\cite{massich2010lesion}
              &j\cite{Shan:2012p14347}
              &k\cite{Yeh:2009p11985}
              &l\cite{Horsch:2001p6028}
              &m\cite{Gomez:2010p14339}
              &n\cite{Huang:2005p11636}
              &o\cite{Huang:2007p6100}
              &p\cite{Cui:2009p14325}\\
  \\\hline \\
  Dataset size:     & 76   & 20   & 32   & 20   & 42   & 480  & 347    & 352  & 25
                    & 120  & 6    & 400  & 50   & 20   & 118  & 488 \\

  techonlogy used for:  &\\
  \quad detection       & \myCoord{Adetect} & \myCoord{Bdetect} & \myCoord{Cdetect} & \myCoord{Ddetect} & \myCoord{Edetect} 
                        & \myCoord{Fdetect} & \myCoord{Gdetect} & \myCoord{Hdetect} & \myCoord{Idetect} & \myCoord{Jdetect}
                        & \myCoord{Kdetect} & \myCoord{Ldetect} & \myCoord{Mdetect} & \myCoord{Ndetect} & \myCoord{Odetect}
                        & \myCoord{Pdetect}\\

  \quad segmetnation    & \myCoord{Aseg} & \myCoord{Bseg} & \myCoord{Cseg} & \myCoord{Dseg} & \myCoord{Eseg} 
                        & \myCoord{Fseg} & \myCoord{Gseg} & \myCoord{Hseg} & \myCoord{Iseg} & \myCoord{Jseg}
                        & \myCoord{Kseg} & \myCoord{Lseg} & \myCoord{Mseg} & \myCoord{Nseg} & \myCoord{Oseg}
                        & \myCoord{Pseg}\\

  \quad post-processing & \myCoord{App} & \myCoord{Bpp} & \myCoord{Cpp} & \myCoord{Dpp} & \myCoord{Epp} 
                        & \myCoord{Fpp} & \myCoord{Gpp} & \myCoord{Hpp} & \myCoord{Ipp} & \myCoord{Jpp}
                        & \myCoord{Kpp} & \myCoord{Lpp} & \myCoord{Mpp} & \myCoord{Npp} & \myCoord{Opp}
                        & \myCoord{Ppp}\\
  \\\hline \\
  \ac{aov} (in \%): & 88.1 & 86.3 & 88.3 & 85.2 & 62.0 & 75.0 & 84.0   & 54.9 & 64.0
                    & 83.1 & 73.3 & 73.0 & 85.0 & 78.6 & 77.6 & 74.5\\
\end{tabular}


\begin{tikzpicture}[remember picture, overlay]
  % single step ACM
  \foreach \x in {Cseg, Cpp, Dpp, Epp, Npp, Opp, Ppp}
  \node[nodeBase, acmStyle, anchor=center] at (\x) {};

  % single step ML
  \foreach \x in {Edetect, Gdetect, Idetect, Pseg}
  \node[nodeBase, mlStyle, anchor=center] at (\x) {};

  % single step Other
  \foreach \x in {Eseg, Iseg, Ipp, Jpp, Kseg, Lseg, Mseg, Mpp, Nseg}
  \node[nodeBase, otherStyle, anchor=center] at (\x) {};

  \node[nodeBase, acmStyle, fit= (Adetect) (Aseg) (App)] {};
  \node[nodeBase, acmStyle, fit= (Bseg)(Bpp)] {};
  \node[nodeBase, mlStyle, fit= (Ddetect)(Dseg)] {};
  \node[nodeBase, mlStyle, fit= (Fdetect)(Fseg)] {};
  \node[nodeBase, mlStyle, fit= (Gseg)(Gpp)] {};
  \node[nodeBase, mlStyle, fit= (Hseg)(Hpp)] {};
  \node[nodeBase, mlStyle, fit= (Jdetect)(Jseg)] {};
  \node[nodeBase, otherStyle, fit= (Odetect)(Oseg)] {};
\end{tikzpicture}
}
    %\caption{\ac{bus} images lesion segmentation strategies compiled from the bulk of the literature: reported quantitative results and methodology highlights.}
    \label{fig:surveyResults:survey}
  \end{subfigure}
  \begin{subfigure}[b]{\textwidth}
    {\tiny \begin{tikzpicture}[scale=.85]


  \def\labels{ c, b, a, p, o, n, m, l, k, j, i, h, g, f, e, d}
  
  \def\reward{88.3,86.3,88.1,74.5,77.6,78.6,85.0,73.0,73.3,83.1,64.0,54.9,84.0,75.0,62.0,85.2}
  \def\dbSize{32,20,76,488,118,20,50,400,6,120,25,352,347,480,42,20}
  \def\dbClass{1,1,2,3,2,1,2,3,1,2,1,3,3,3,1,1}		
  \def\cZoom{3} 
  \def\percentageLabelAngle{90}
  \def\nbeams{16}
  \pgfmathsetmacro\beamAngle{(360/\nbeams)}
  \pgfmathsetmacro\halfAngle{(180/\nbeams)}
  % \def\globalRotation{10}
  \pgfmathsetmacro\globalRotation{\halfAngle}

  % draw manual AOV results
  \filldraw[aov!15!white,even odd rule] (0,0) circle [radius={\cZoom*.852}] (0,0) circle [radius={\cZoom*.8}];
  \draw[thin,color=aov!50!white,dashed] (0,0) circle [radius={\cZoom*.852}] (0,0) circle [radius={\cZoom*.8}];

  % \foreach \x in {.125,.25, ...,1} { \draw[thin]  (0,0) circle [radius={2*\x}]; }
  % draw the radiants with the reference label
  \foreach \n  [count=\ni] in \labels
  {
    \pgfmathsetmacro\cAngle{{(\ni*(360/\nbeams))+\globalRotation}}
    \draw [thin] (0,0) -- (\cAngle:{\cZoom*1}) ;
    \draw	(\cAngle:{\cZoom*1.1})  node[fill=white, inner sep=0pt] {{\tiny \textbf  \n}}; %referencies
  }

  % draw the % rings 
  \foreach \x in {12.5,25, ...,100} 
  \draw [thin,color=gray!50] (0,0) circle [radius={\cZoom*\x/100}];

  \foreach \x in {50,75,100}
  { 
    \draw [thin,color=black!50] (0,0) circle [radius={\cZoom/100*\x}];
    \foreach \a in {0, 180} \draw ({\percentageLabelAngle+\a}:{\cZoom*0.01*\x}) node  [inner sep=0pt,outer sep=0pt,fill=white,font=\fontsize{5}{5}\selectfont]{$\x$};
  }
  
  %% draw our results ring
  \draw [thick,color=black] (0,0) circle [radius={\cZoom*.63}];
  \foreach \a in {0, 180} \draw ({\percentageLabelAngle+\a}:{\cZoom*0.63}) node  [inner sep=0pt,outer sep=0pt,fill=white,font=\fontsize{5}{5}\selectfont]{$62.3$};


  % draw the path of the percentages
  \def\aux{{\reward}}
  \pgfmathsetmacro\origin{\aux[\nbeams-1]} 
  \draw [aov, thick] (\globalRotation:{\cZoom*\origin/100}) \foreach \n  [count=\ni] in \reward { -- ({(\ni*(360/\nbeams))+\globalRotation}:{\cZoom*\n/100}) } ;

  % label all the percentags
  \foreach \n [count=\ni] in \dbSize 
  {
    \pgfmathsetmacro\cAngle{{(\ni*(360/\nbeams))+\globalRotation}}
    \pgfmathsetmacro\nreward{\aux[\ni-1]}
%    \draw (\cAngle:{\cZoom*1.4}) node[align=center] {{\color{aov}\nreward $\%$} \\ {\color{db}\n} };
  } ;

  % draw the database rose
  \def\dbScale{\09}
  \foreach \n [count=\ni] in \dbClass
  \filldraw[fill=db!20!white, draw=db!50!black]
  (0,0) -- ({\ni*(360/\nbeams)-\halfAngle+\globalRotation}:{\cZoom*\n/9}) arc ({\ni*(360/\nbeams)-\halfAngle+\globalRotation}:{\ni*(360/\nbeams)+\halfAngle+\globalRotation}:{\cZoom*\n/9}) -- cycle;
  \foreach \x in {1,2,3}
  \draw [thin,color=db!50!black,dashed] (0,0) circle [radius={\cZoom*\x/9}];

  %% draw the legend
  \node [anchor=north west,yshift=-6pt] at (-45:\cZoom*1.1){
    \begin{tikzpicture}[font=\tiny]
      \begin{customlegend}[ 
        legend style={ draw=none},
        legend entries={AOV, Our results}
        ]
        \addlegendimage{db,draw=aov,thick,sharp plot}
        \addlegendimage{db,draw=black,thick,sharp plot}
      \end{customlegend}
    \end{tikzpicture}};

  \node [anchor=north east, xshift=-5pt] at (-130:\cZoom*1.1){
    \begin{tikzpicture}[font=\tiny]
      \begin{customlegend}[ 
        legend style={ draw=none},
        legend entries={Dataset, Experts\cite{gerard2013}}
        ]
        \addlegendimage{db,fill=db!20!white, draw=db!60!gray, ybar, ybar legend}
        \addlegendimage{db,fill=aov!15!white,draw=aov,dashed,area legend}
      \end{customlegend}
    \end{tikzpicture}};

  %% draw the domain of each class 
  %\def\puta{	3/0/{\tikz{\node[nodeBase,acmStyle]{};}ACM},
  \def\puta{	3/0/{ACM},
    3/3/{ACM+Other},
    3/6/{Other}}
  \def\putaa{  	2/9/{Other+ML},
    3/11/{ML},
    2/14/{ML+ACM}}

  \foreach \numElm/\contadorQueNoSeCalcular/\name [count=\ni] in \puta
  {

    \pgfmathsetmacro\initialAngle{(\contadorQueNoSeCalcular*\beamAngle)+\halfAngle+\globalRotation}
    \pgfmathsetmacro\finalAngle  {((\numElm+\contadorQueNoSeCalcular)*\beamAngle)+\halfAngle+\globalRotation}
    \pgfmathsetmacro\l  {\cZoom*1.1+.3pt}
    \draw (\initialAngle:{\cZoom*1.1}) -- (\initialAngle:{\cZoom*1.1});
    \draw [ |<->|,>=latex] (\initialAngle:\l) arc (\initialAngle:\finalAngle:\l) ;
    \pgfmathsetmacro\r  {\cZoom*1.1+.45pt}
    {\draw [decoration={text along path,  text={\name},text align={center}},decorate] (\finalAngle:\r) arc (\finalAngle:\initialAngle:\r);}
  }
  
  \node[nodeBase,acmStyle] at (63:\cZoom*1.28) {};
  \node[nodeBase,mlStyle] at (-61:\cZoom*1.28) {};
  \node[nodeBase,otherStyle] at (-161.5:\cZoom*1.28) {};

  \foreach \numElm/\contadorQueNoSeCalcular/\name [count=\ni] in \putaa
  {

    \pgfmathsetmacro\initialAngle{(\contadorQueNoSeCalcular*\beamAngle)+\halfAngle+\globalRotation}
    \pgfmathsetmacro\finalAngle  {((\numElm+\contadorQueNoSeCalcular)*\beamAngle)+\halfAngle+\globalRotation}
    \pgfmathsetmacro\l  {\cZoom*1.1+.3pt}
    \draw (\initialAngle:{\cZoom*1.1}) -- (\initialAngle:{\cZoom*1.1});
    \draw [ |<->|,>=latex] (\initialAngle:\l) arc (\initialAngle:\finalAngle:\l) ;    									 
    \pgfmathsetmacro\r  {\cZoom*1.1+.61pt}
    {\draw [decoration={text along path, text={\name},text align={center}},decorate] (\initialAngle:\r) arc (\initialAngle:\finalAngle:\r);}    			 
  }
  
  \node[] at (5.8,3) 
  {\tiny \newcommand{\myCoord}[1]{
  \tikz[remember picture]{\coordinate[remember picture] (#1) at (0,0);
    %\fill[red] (#1) circle[radius=1pt];
  }
}


\begin{tikzpicture}[remember picture]
  \tikzstyle{noMargin} = [inner sep=0mm, outer sep=0mm]
  %\node[draw, noMargin, remember picture, anchor=north west,
  %minimum width = 3cm,
  %label={[label distance=-0.8cm,text depth=15pt,rotate=90]left:description}
  %](methodTech){
  %  \qquad
  %  \begin{tabular}{lll}
  %    \multicolumn{3}{l}{technology used for:}\\ 
  %    \multicolumn{2}{l}{\qquad detection}& \myCoord{mlInit}\\
  %    \multicolumn{2}{l}{\qquad segmentation}\\
  %    \multicolumn{2}{l}{\qquad post-processing}&\myCoord{mlFin}\\
  %    $\mathcal{S}$: & \multicolumn{2}{l}{Quick-Shift super-pixels}\\
  %    $\arg \min (U(\cdot))$:&\multicolumn{2}{l}{\acl{gc}}\\
  %    $D(\cdot)$:&\multicolumn{2}{l}{\tikz[noMargin, baseline=(img.north)]{\node[noMargin](img){\includegraphics[width=1cm]{vcues}};}}\\
  %    &\multicolumn{2}{l}{\acs{rbf}-\acs{svm}}\\
  %    $V(\cdot,\cdot)$:&\multicolumn{2}{l}{Homogenity}\\
  %  \end{tabular}
  %};
  %\node[yshift=2pt,remember picture, overlay, nodeBase, mlStyle, fit= (mlInit) (mlFin)]{};
\node[draw, %below= 2pt of methodTech,
minimum width = 3.9cm,
  label={[label distance=-0.5cm,text depth=15pt,anchor=south, rotate=90]left:testing}
  ](testingNode){
    \begin{tabular}{lclclc}
      \multicolumn{2}{l}{Database size:} & \multicolumn{4}{l}{$16$} \\
      \multicolumn{2}{l}{\ac{gt}:}       & \multicolumn{4}{l}{multi-label} \\
      \multicolumn{2}{l}{Task:}          & \multicolumn{4}{l}{$\mathcal{L} = \{\text{lesion}, \overline{\text{lesion}}\}$} \\
      \multicolumn{2}{l}{Trainning:}     & \multicolumn{4}{l}{Leave-one-Patient-Out}\\
      \ac{aov}:      & .623                                                                               & \acs{fpr}: & .4 & \acs{fnr}: & .008
    \end{tabular}
    };
%  \node[
%        draw=red,
%        minimum width=\textwidth,
%        fit=(current bounding box.north west) (current bounding box.south east),
%      ]at (current bounding box.center){};
\end{tikzpicture}

};
\end{tikzpicture}
 }
    %\caption{{\small comparison}}
    \label{fig:surveyResults:comparison}
  \end{subfigure}
  \caption{Quantitative results compilation and comparison}
  \label{fig:surveyResults}
\end{figure}


%%% Local Variables: 
%%% mode: latex
%%% TeX-master: "../../master.tex"



\section{Conclusions}

\bibliography{./content/literature_review}   %>>>> bibliography data in report.bib
\bibliographystyle{splncs03}

\end{document} 
