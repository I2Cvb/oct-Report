% % include the figures path relative to the master file
% \graphicspath{ {./content/results/figures/} }

\section{Experiments and Validation }
\subsection{Data sets}
\subsubsection{SERI} - data sets were acquired by Singapore Eye Research Institute (SERI), using CIRRUS TM (Carl Zeiss Meditec, Inc., Dublin, CA) \ac{sdoct} device. The data sets consist of 32 \ac{oct} volumes (16 \ac{dme} and 16 normal cases). Each volume contains 128 B-sane with  dimension of 512 $\times$ 1024 pixels.  All \ac{sdoct} images are read and assessed by trained graders and identifies as normal or \ac{dme} cases based on evaluation of retinal thickening, hard exudates, intraretinal cystoid space formation and subretinal fluid.


\subsubsection{Duke} - data sets published by Srinivasan et al. \cite{Srinivasan2014} were acquired in Institutional Review Board-approved protocols using Spectralis \ac{sdoct} (Heidelberg Engineering Inc., Heidelberg, Germany) imaging at Duke University, Harvard University and the University of Michigan. This data sets consist of 45 \ac{oct} volumes (15 \ac{amd}, 15 \ac{dme} and 15 normal). In this study we only consider a subset of the original data containing 15 \ac{dme} and 15 normal \ac{oct} volumes.

\subsection{Validation}
For evaluation purposes, the results have been cross-validated, by splitting the data in training and testing using a \ac{lopo} strategy. In this manner for each round a pair \ac{dme}, normal has been selected to be used as the round test set, while the rest of the data sets have been used as a training. Doing the cross validation in this manner, has the limitation that despite the fact that the results are robust due to the cross validation, no results variance can be reported. However, and despite this limitation, \ac{lopo} has been choose due to the reduced amount of OCT volumes available.

\subsection{Experiment}
The SERI data sets are provided in complete \ac{oct} volumes by 512$\times$1024$\times$128 dimensions. Using this data sets, first the three low-level features such as \ac{lbp}, \ac{lbp}+\ac{pca} and \ac{lbptop} are extracted. The rotation invariant uniform ($riu2$) descriptors are calculated with the $P$ number of 8, 16 and 24 for the radius if 1, 2 and 3 respectively. The features are classified using RF with 100 tress. Table \ref{tab:LbPTopVolumeResult} shows the relative results for $8riu2$, $16riu2$, $24riu2$ and their combination $8riu2 + 16riu2 + 24riu2$. The results are presented in terms of \ac{se} and \ac{sp} percentages.  

The second experiment is carried out using high-level features and \ac{bow} approach, on SERI data sets. The first high-level feature \ac{lbp}+\ac{bow} is obtained by applying \ac{bow} with 32 visual-words on the previously low-level \ac{lbp} features (applied on each B-scan). The second and third high-level descriptors are obtained using a dense approach by applying the \ac{sw} of size (7$\times$7) on each B-scan and \ac{sw} of size (7$\times$7$\times$7) to the whole volume respectively. \ac{lbp}+\ac{bow}+\ac{sw} represent the second high-level feature where the 2D-\ac{lbp} features are extracted for each sliding window on each B-scan and the visual-words are selected from the pool, consisting of their histograms. The third high-level feature, \ac{lbptop}+\ac{bow}+\ac{sw}, is defined using \ac{lbptop}. By using the sliding window the 3D-\ac{lbp} features are extracted for each patch. Same as previous experiment with low-level features, the descriptors are calculated with the $P$ number of 8, 16 and 24 for the radius if 1, 2 and 3 respectively. The obtained results of this experiment are illustrated in Tab. \ref{tab:SERIBoWResult}. 

In order to compare our proposed framework the third experiment is carried out using the subsection of Duke data sets \cite{Srinivasan2014}. The OCT volumes provided by this data sets are of different volume size, cropped and denoised by the method of authors choice. Subsequently only the second experiment with high-level features and low-level \ac{lbptop} features comply with these requirements. The number of visual-words and the size of \ac{sw} for 2D and 3D features are the same than the previous experiment. The 2D and 3D \ac{lbp} features are extracted with $P$ number of 8, 16 and 24 for the radius if 1, 2 and 3 respectively. The obtained results for this experiment are shown in Tab. \ref{tab:DukeBoWResult}.
%----------

%%% Local Variables: 
%%% mode: latex
%%% TeX-master: "../../master"
%%% End: 
