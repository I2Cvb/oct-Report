\section{This generic framework applied to \ac{bus} imaging}\label{sec:method:dterm:feat}

\begin{figure}[t]
    \centering
    \hspace{\fill}
    \begin{subfigure}[b]{0.30\textwidth}
        \centering
        \begin{tikzpicture}
          \node[inner sep=0, draw] (breast) {\includegraphics[width=0.4\textwidth]{breast}};
          \node[inner sep=0, draw, below= 2pt of breast] (slice) {\includegraphics[width=0.4\textwidth]{slice}};
        \end{tikzpicture}
        %\caption{{\small Breast structure}}    
        \label{fig:features:breast}
    \end{subfigure}
    \hfill
    \begin{subfigure}[b]{0.34\textwidth}   
        \centering 
        \includegraphics[trim = 65 345 200 124, clip,width=0.9\textwidth]{birads}
        %\caption[]%
        %{Breast lesion characteristics in \ac{us} screening influencing clinical management~\cite{biradsus}}    
        \label{fig:features:lexicon}
    \end{subfigure}
    \hspace{\fill}
    \caption {{\footnotesize Visual reference: (a) breast structures, (b) US BI-RADS lexicon.}} 
    \label{fig:features}
\end{figure}

In this section, the generic framework presented in the previous section is applied to \ac{bus} imaging. We need to define our problem properly. We recall that the aim in segmentation is to affect to a discrete set of elements $\mathcal{S}$, a label $l$ from a labelling set $\mathcal{L}$. In our case, our labelling set $\mathcal{L} = \{ \text{chest wall}, \text{lungs}, \dots, \text{lesion} \}$ (see Fig\,\ref{fig:methodTerms:problem} for the entire set of labels).

As illustrated in Fig.\,{\color{red}??}, choices have to be made regarding the different elements: the set $\mathcal{S}$, the data term $D(\cdot)$, the pairwise term $V(.)$, and the optimizer choice. These preferences are summarized in Table~{\color{red}??} and justified thereafter.

$\mathcal{S}$ is chosen to be the result from an over-segmentation of the input image using the Quick-shift super-pixel. The structures of the breast and their rendering when using a hand-held 2D \ac{us} probe are sketched in Fig.\,\ref{fig:features}. Figure~\ref{fig:features:lexicon} illustrates the lexicon proposed by the \ac{acr}~\cite{biradsus} and used by clinicians to perform their diagnosis. Thus, our aim is to generate a set of computer vision features which is able to encode the characteristic described in the lexicon. The selected features are the following:

\begin{description}
  \item[Appearance] 
    Based on the multi-labelled \ac{gt}, a \ac{mad} histogram model for every tissue label is built. The Appearance feature is computed as the $\chi^2$ distance between a histogram of $s$ and the models generated.
  \item[Atlas] 
    Based on the multi-labelled \ac{gt} an atlas is build to encode the labels likelihood based on the location of $s$.
  \item[Brightness] 
    Intensity descriptors are computed based on statistics of $s$ (\emph{i.e:} mean, median, mode) and  are compared with some intensity markers of the set $\mathcal{S}$ such as the minimum intensity value, the maximum, its mean, etc.
  \item[\ac{sift}-\ac{bof}]
    $s$ is described as an histogram of visual words based on \ac{sift}~\cite{massich2014sift}. The dictionary is built with $36$ words.
\end{description}

The relationships between the lexicon and the descriptors previously described is depicted in Table~{\color{red}??}. More precisely, we highlight the corresponding elements of the lexicon which is encoded by each feature. A choice regarding the encoding of the data term $D(\cdot)$ has to be made by using a \ac{ml} classifier. An \ac{svm} classifier with an \ac{rbf} kernel is selected to determine the data model during the training stage. The pairwise term is our framework was defined as in Eq.\,\eqref{eq:smoothing}. The optimization method used as solver to minimize our cost function $U(\cdot)$ is \ac{gc}. \ac{gc} when applicable allows to rapidly find a strong local minima guaranteeing that no other minimum with lower energies can be found~\cite{delong2012fast}. \ac{gc} is applicable if, and only if, the pairwise term favours coherent labelling configurations and penalizes labelling configurations where neighbours labels differs; such is our case, given by Eq.\,\eqref{eq:smoothing}.

%%% Local Variables: 
%%% mode: latex
%%% TeX-master: "../../master"
%%% End: 
