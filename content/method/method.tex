% include the figures path relative to the master file
\graphicspath{ {./content/method/figures/visual_cues/}{./content/method/figures/}}

\section{Description of the segmentation methodology}\label{sec:method}

Optimization methodologies offer a standardized manner to approach segmentation by minimizing an application-driven cost function~\cite{cremers2007review}.
Figure~\ref{fig:method} illustrates a generic representation of the segmentation strategy, whereas concrete examples of its terms can be found in \cref{sec:methodApp} applied to \ac{bus}.
The overall segmentation can be seen as a three-steps strategy: 
(1) a mapping of the image into a discrete set of elements $\mathcal{S}$, 
(2) the optimization stage which is formulated as a \emph{metric labelling} problem, 
and (3) a re-mapping the labels obtained from the previous stage to produce the final delineation. 

\begin{figure}[htpb]
  \centering
  \includegraphics[width=0.9\linewidth]{method}
  \caption{Conceptual block representation of the segmentation methodology.
    %\footnote{\cref{fig:methodterms} illustrates the $\mathcal{S}$, $D(\cdot)$, and $V(\cdot)$ for the applied case of delineating breast structures in \ac{us} data.}
    %\footnote{(todo:add all the names of the elements in the figure)}
  }
  \label{fig:method}
\end{figure}


In order to formulate the segmentation like a metric labelling problem, the image is conceived as a discrete set of elements $\mathcal{S}$ that need to be labelled using a label $l$ from the labelling set $\mathcal{L}$.
Let $\mathcal{W}$ be all the possible labelling configurations of the set $\mathcal{S}$, given $\mathcal{L}$.
Let $U(\cdot)$ be a cost function encoding the goodness of the labelling configuration $\omega \in \mathcal{W}$ based on the appearance of the elements in $\mathcal{S}$, their inner relation and some designing constraints.
Then, the desired segmentation $\hat{\omega}$ corresponds to the labelling configuration that minimizes this cost function, as described in Eq.\,\eqref{eq:costMin}.

\begin{equation}
\hat{\omega} = \arg \min_{\substack{\omega}} \,U(\omega)
\label{eq:costMin}
\end{equation}


This goodness measure $U(\cdot)$ must be defined to take into account the appearance of the target region, its relation with other regions and other designing constraints.
Equation~\eqref{eq:labelingeq} describes this cost function as the combination of two independent costs that need to be simultaneously minimized as a whole.

\begin{equation}
  U(\omega) = \sum_{s\in \mathcal{S}} D_s(\omega_s) + \sum_{s \in \mathcal{S}}\sum_{r \in \mathcal{N}_{s}} V_{s,r}(\omega_s,\omega_r)
  \label{eq:labelingeq}
\end{equation}

Where, the left hand side of the expression integrates the so-called \emph{data} term, while the right hand side integrates the \emph{pairwise} term, which is also referred as the \emph{smoothing} term.
Both terms are shaped by $\mathcal{S}$ and evaluated in the labelling space $\mathcal{W}$.

{\color{red}In our quest to optimize the cost function $U(\cdot)$, it is required to define a representation for the set $\mathcal{S}$, a data term $D(\cdot)$, a pairwise term $V(\cdot)$, and a proper minimization methodology.}

\paragraph{The set $\mathcal{S}$}
that represents the image can be, in general, any discrete set representing the image (i.e.\, pixels, overlapping or non overlapping windows, super-pixels, etc.). 

\paragraph{The data term} \label{sec:method:dataTerm}
, given a label configuration $\omega \in \mathcal{W}$, penalizes the labelling of a particular image element or site ($\omega_s = l$) based on the data associated to $s$.
In this manner, $D_s(\omega_s=l_\cmark) << D_s(\omega_s=l_\xmark)$.

Designing an obscure heuristic to comply with the desired behaviour of $D(\cdot)$ out of the box, is rather a complicated task.
{\color{blue}
Therefore, an easier and cleaner approach is to take advantage of \ac{ml} techniques to design this data cost in a systematic manner based on a training stage. 
The idea is to generate a data model for each label (or class) in $\mathcal{L}$ from training samples, and let $D(\cdot)$ be a distance or goodness measure reflecting the likelihood for $s$ to belong to class $l$.
Different features to represent the data, custom construction of the data models by using different classifiers, training techniques or including arbitrary constraints; can be used to achieve the desired data term without changing the overall scheme.}

{\color{red} Defining the data term $D(\cdot)$ with the help of \ac{ml} follows a systematic process that is flexible enough to encode any desired behaviour based on a training stage.  This representation is in fact depicted in the upper row in Fig.\,\ref{fig:method}.  For each site $s \in \mathcal{S}$, features describing $s$ are designed. Then, different optional steps can be applied to this set of feature: (i) features normalization, (ii) features selection or (iii) features extraction. Finally, the data term $D(\cdot)$ is encoded based on \ac{ml} classifiers, the features and a training step. }

\paragraph{The pairwise term} \label{sec:method:mrfTerm}
%The pairwise term 
represents the cost associated to $\omega_s$ taking into account the labels of its neighbour sites, $\omega_r$, $r \in \mathcal{N}_{s}$. 
This term models a \ac{mrf} or a \ac{crf}.
The typical form of this term, given in Eq.\,\eqref{eq:smoothing}, is called homogenization which acts as a regularization factor favouring configurations that have a coherent labelling.

\begin{equation}
V_{s,r}(\omega_s,\omega_r) = 
\begin{cases}
    \beta, & \text{if } \omega_s \ne \omega_r\\
    0,              & \text{otherwise}
\end{cases}
\label{eq:smoothing}
\end{equation}


% Review %contribute have different or variable costs (see \cref{fig:methodTerms:boundary}) are also possible by taking into account not only relations in $\mathcal{S}$ of but also image information (see \cref{fig:method}). 
%Further details can be found in \cref{sec:smoothing}.

\paragraph{The minimization strategy} \label{sec:method:min}
is determined by the nature of $U(\cdot)$ and $\mathcal{W}$, since not all the minimization strategies are applicable or adequate to find $\hat{\omega}$.
The size of the labelling space $|\mathcal{W}|=|\mathcal{L}|^{|\mathcal{S}|}$, discontinuities in $U(\omega)$ due to $\mathcal{W}$ or the problem of local minima, 
along with all the particular constrains of all the different minimization methodology.
Need to be taken into account while choosing the most desirable minimization strategy.

%%% Local Variables: 
%%% mode: latex
%%% TeX-master: "../../master.tex"
